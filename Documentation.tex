\documentclass{article}

\begin{document}
\section{Entwicklungsumgebung}
Verwendet wurde javac in der version 20.0.1
und openjdk version 20.0.1+9-29 mit 64-bit Server VM.
Weitere Verwendete Programme sind:
\begin{enumerate}
\item GNU Emacs
\item Visual Studio
\item XeTeX
\item git
\end{enumerate}
\section{Benutzung}
Jeder Kontakt ist entweder eine Person oder ein Unternehmen.
Eine Person als Kontakt kann einen Vornamen, einen Nachnamen, einen Wohnort,
eine PLZ des Wohnortes, eine Straße, eine Hausnummer haben.
Ein Unternehmen kann einen Namen, eine Ort des Hauptsitzes, eine PLZ des Ortes
des Hauptsizes, einen Straße des Hauptsitzes, eine Hausnummer des Hauptsitzes
haben, sowie einen Vornahmen des Besizers und einen Nachnahmen des Besitzers
haben. Alle diese Attribute können in beliebiger Kombination abwesend sein. 
Um das Addressbuch zu benutzen, muss eine ein Addressbook (die Klasse)
instanziert werden, auf dieser Instanz muss zunächst \textit{.addContact}
aufgerufen werden. Pro Aufruf kann ein Kontakt hinzugefügt werden.
Anschließend können die hinzugefügten Kontakte mit \textit{.printContacts}
ausgedruckt werden oder mit \textit{search(string)} durchsucht werden, ob
welche Kontakte den searchterm \textit{string} in irgendeinem Feld
enthalten. (Es kann auch nach Zahlen gesucht werden).
Weiterhin kann über \textit{.deleteContact()} ein Kontakt eliminiert werden.
\end{document}
